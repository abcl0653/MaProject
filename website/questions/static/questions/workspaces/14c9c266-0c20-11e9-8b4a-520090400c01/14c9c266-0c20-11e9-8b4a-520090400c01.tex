%!TEX program = xelatex

\documentclass[varwidth,convert]{BHCexam1_simple}
\usepackage{gensymb}
\usepackage{mathrsfs}
\usepackage{tikz,pgfplots,float} %绘图
\usepackage{tkz-euclide}
\usepackage{graphicx}
\usepackage{wrapfig}
\usepackage[english]{babel}
\usepackage{caption}
\usepackage{CJK}
\usepackage{enumitem}
\usetikzlibrary{calc,quotes,angles,babel,intersections,arrows,automata,positioning}

\pgfplotsset{compat=1.15}
\tikzset{ 
  right angle quadrant/.code={ 
   \pgfmathsetmacro\quadranta{{1,1,-1,-1}[#1-1]} % Arrays for selecting quadrant 
   \pgfmathsetmacro\quadrantb{{1,-1,-1,1}[#1-1]}}, 
  right angle quadrant=1, % Make sure it is set, even if not called explicitly 
  right angle length/.code={\def\rightanglelength{#1}}, % Length of symbol 
  right angle length=2ex, % Make sure it is set... 
  right angle symbol/.style n args={3}{ 
   insert path={ 
   let \p0 = ( $(#1)!(#3)!(#2)$ ) in % Intersection 
    let \p1 = ( $(\p0)!\quadranta*\rightanglelength!(#3)$ ), % Point on base line 
    \p2 = ( $(\p0)!\quadrantb*\rightanglelength!(#2)$ ) in % Point on perpendicular line 
    let \p3 = ( $(\p1)+(\p2)-(\p0)$ ) in % Corner point of symbol 
   (\p1) -- (\p3) -- (\p2) 
   } 
  } 
}

\setlength{\textwidth}{20cm}

\begin{document}

%%% -------- Anchor Start -------- %%%

如图,直线~$AB,CD$~相交于点~$O$,已知$\angle AOD=130\degree$,$\angle AOC:\angle EOC=5:4$,求$\angle BOE$的大小.

\hfill
\begin{tikzpicture}[scale=0.8]
\coordinate[label=below left:$O$] (O) at (0,0);
\coordinate[label=below:$A$] (A) at (-3,0);
\coordinate[label=below:$B$] (B) at (3,0);
\coordinate[label=above:$E$] (E) at (0,3);
\draw (A) -- (B);
\draw (O) -- (E);
\begin{scope}[rotate=-50,shift={(-1,0)}]
\coordinate[label=above left:$C$] (C) at (-3,0);
\coordinate[label=above right:$D$] (D) at (3,0);
\draw (C) -- (D);
\end{scope}
\end{tikzpicture}


%%% -------- Anchor End ---------- %%%

\end{document}